\documentclass[a4paper]{amsart}
\usepackage{fullpage}
\usepackage{amsmath}
\usepackage{amsfonts}
\usepackage{amssymb}
\usepackage{amsthm}
\usepackage[utf8]{inputenc}
\usepackage{graphics}
\usepackage{algpseudocode}
\usepackage{algorithm}
\usepackage{tikz}
\usepackage{t1enc}
\usepackage{mathrsfs}
\usepackage{enumerate}
\usepackage{url}
\def\magyarOptions{defaults=hu-min}
\usepackage[magyar]{babel}
\usepackage{enumitem}
\newlist{Axiom}{enumerate}{10}
\setlist[Axiom]{label={(AX\arabic*)},ref=AMon\arabic*,leftmargin=*}
\newcommand{\adj}{\mathop{\mathrm{adj}}}
\newcommand{\GL}{\mathop{\mathrm{GL}}}
\newcommand{\Sym}{\mathop{\mathrm{Sym}}}
\newcommand{\id}{\mathop{\mathrm{id}}}
\newcommand{\stab}{\mathop{\mathrm{Stab}}}
\newcommand{\orb}{\mathop{\mathrm{Orb}}}
\newcommand{\im}{\mathop{\mathrm{im}}}
\newcommand{\aut}{\mathop{\mathrm{Aut}}}
\newcommand{\inn}{\mathop{\mathrm{Inn}}}
\newcommand{\cl}{\mathop{\mathrm{Cl}}}
\newcommand{\ded}{\mathop{\mathrm{ded}}}
\newcommand{\On}{\mathop{\mathrm{On}}}
\newtheorem{lem}{lemma}[section]
\newtheorem{exc}{feladat}
\newtheorem{theo}{tétel}[section]
\newtheorem{theos}{tétel*}
\newtheorem{state}{állítás}[section]
\newtheorem{exmpl}{példa}[section]
\newtheorem{fairytale}{mese}[section]
\newtheorem{col}{következmény}[section]
\newtheorem{megj}{megjegyzés}[section]
\linespread{1.3}
\newtheorem{defi}{definíció}[section]
\newenvironment{sol}[1][Megoldás.]{\begin{trivlist}
\item[\hskip \labelsep {\bfseries #1}]}{\end{trivlist}}
\newenvironment{comm}[1][Megjegyzés.]{\begin{trivlist}
\item[\hskip \labelsep {\bfseries #1}]}{\end{trivlist}}
\begin{document}
\title{A matematika alapjai}
\begin{abstract}
mi is ez..
\end{abstract}
\maketitle
\part*{1. logika blokk}
\section{előadás}
Később...
\section{előadás}
\subsection{A $0$-rendű logika szemantikája}
\begin{defi}[interpretáció, kiértékelés]{A $k:\lbrace \text{ítéletváltozók} \rbrace \rightarrow \lbrace i, h \rbrace$} függvényt interpretációnak vagy változókiértékelésnek nevezzük.
\end{defi}
\subsubsection{A logikai műveletek jelentése}
\begin{displaymath}
\begin{array}{|c|c||c|c|c|c|c}
   x
 & y
 & \lnot{}x
 & x\land{}y
 & x\lor{}y
 & x\Rightarrow y
 & x\Leftrightarrow{}y \\
\hline
i & i & h & i & i & i & i \\
i & h & h & h & i & h & h \\
h & i & i & h & i & i & h \\
h & h & i & h & h & i & i \\
\hline
\end{array}
\end{displaymath}
\subsubsection{Formulák jelentése}
\begin{defi}[Az igazság definíciója]{Legyen $k$ egy értékelés $\varphi$ pedig egy $0$-rendű formula. Jelölje $k \models \varphi$ azt, hogy $\varphi$ igaz $k$-ban ($k$ értékelés mellett.) 
$$k \models \varphi, \text{ ha } 
\begin{cases}
k(\varphi) = i, & \varphi \text{ ítéletváltozó} \\
k \not\models \psi, & \varphi = \lnot \psi \\
k \models \psi_1 \text{ és } k \models \psi_2, & \varphi = \psi_1 \land \psi_2 \\
... & ...
\end{cases}$$}
\end{defi}
A kipontozott részek a 2.1.1-ben lévő táblázat szerint folytatódnak.
\begin{megj}
Összetett formula jelentése (igazsága) kiszámolható a részei jelentéséből (igazságából).
\end{megj}
\subsection{Az 1-rendű formulák szemantikája}
\begin{defi}[1-rendű struktúra]
$\mathcal{A} = (A,f_i^\mathcal{A},R_j^\mathcal{A},C_k^\mathcal{A})_{i \in I, j \in J, k \in K}$, ahol $A$ nemüres alaphalmaz és minden $i \in I$, $j \in J$ és $k \in K$-ra: $f_i^{\rho(i)} : A \rightarrow A$ egy $\rho(i)$-változós függvény, $R_j \subseteq A^{\delta(j)}$ egy $\delta(j)$-változós reláció, valamint $C^{\mathcal{A}}_k \in A$ egy konstans. $I, J, K$ tetszőleges halmaz. Egy struktúra típusát $(I,J,K,\rho,\delta)$-val adjuk meg.
\end{defi}

\begin{exmpl}
\begin{minipage}[t]{\linewidth}
\begin{enumerate}
\item Minden csoport (gyűrű, háló) struktúra.
\item Minden részbenrendezett halmaz struktúra.
\item Minden gráf struktúra.
\item $\mathcal{N} := ( \mathbb{N}, +, \cdot, =, \leqslant, 0,1)$ egy struktúra.
\end{enumerate}
\end{minipage}
\end{exmpl}

\begin{defi} Legyen $L$ egy $1$-rendű nyelv. Az $\mathcal{A}$ struktúra az $L$ nyelv egy modellje, ha $\mathcal{A}$ és $L$ típusa azonos.
\end{defi}
\begin{defi}Legyen $\mathcal{A}$ az $L$ nyelv modellje. Egy $k: \lbrace \text{változók} \rbrace \rightarrow \mathcal{A}$ függvényt az $\mathcal{A}$ feletti (ki)értékelésnek nevezzük.
\end{defi}
\begin{defi}[Termek jelentése] Legyen $k$ egy $\mathcal{A}$ feletti értékelés és $t$ egy term. Ekkor $t$ jelentése $\mathcal{A}$-ban $k$ értékelés mellett
$$t^\mathcal{A}[k] := 
\begin{cases}
k(t), & \text{ ha $t$ változó} \\
c^\mathcal{A}, & \text{ ha $t=c$ konstans} \\
f^{\mathcal{A}}(t^{\mathcal{A}}_1[k], t^{\mathcal{A}}_2[k], \ldots, t^{\mathcal{A}}_n[k]), & \text{ ha $t = f(t_1, t_2, \ldots, t_n)$}
\end{cases}
$$
\end{defi}
\begin{defi} Legyenek $k, k'$ értékelések, $x$ pedig egy változó. Azt mondjuk, hogy $k$ és $k'$ $x$-közel vannak egymáshoz (és azt írjuk, hogy $k \stackrel{x}{\equiv} k'$), ha minden $y$-ra: $x \neq y$ $\Rightarrow$ $k(y)=k'(y)$.
\end{defi}
\begin{defi} Legyen $\varphi$ egy formula, $\mathcal{A}$ struktúra, $k$ pedig egy $\mathcal{A}$ feletti értékelés. Azt mondjuk, hogy $\varphi$ igaz $\mathcal{A}$-ban $k$ értékelés mellett, akkor és csak akkor:
$$
\mathcal{A} \models \varphi[k] = 
\begin{cases}
t^{\mathcal{A}}_1[k] = t^{\mathcal{A}}_2[k], & \text{ ha $\varphi =$ ''$t_1 = t_2$''} \\
(t^\mathcal{A}_1[k], \ldots, t^{\mathcal{A}}_n[k]) \in R^{\mathcal{A}}, & \text{ ha $\varphi = R(t_1, \ldots, t_2)$} \\
\text{Mint $0$-rendben,} & \text{ ha $\varphi =$ ''$\lnot \psi$'', $\varphi =$ ''$\psi_1 \lor \psi_2$'', stb.} \\
\text{Létezik olyan $k \stackrel{x}{\equiv} k'$,}& \text{ ha $\varphi =$ ''$\exists x \psi$'', akkor $\mathcal{A} \models \psi[k']$ } \\
\text{Minden $k' \stackrel{x}{\equiv} k$-ra $\mathcal{A} \models \psi[k']$,} & \text{ ha $\varphi =$ ''$\forall x \psi.$''}
\end{cases}
$$
\end{defi}
\begin{defi}
$\mathcal{A} \models \varphi$, ha minden $\mathcal{A}$ feletti $k$ értékelésre $\mathcal{A} \models \varphi[k]$.
\end{defi}
\begin{defi} Legyen $K$ struktúraosztály, $\Sigma$ pedig formulahalmaz. $K \models \Sigma \Leftrightarrow (\forall \mathcal{A} \in K)(\forall \varphi \in \Sigma)(\mathcal{A} \models \varphi)$.
\end{defi}
\begin{defi} Legyen $\Sigma$ $0$- vagy $1$-rendű formulahalmaz $\varphi$ formula a megfelelő rendben. Azt mondjuk, hogy $\Sigma$-ból következik $\varphi$ és azt írjuk, hogy $\Sigma \models \varphi$, ha
\begin{itemize}
\item $0$-rendben: minden $k$ értékelésre, ha $k \models \Sigma$ ($\Sigma$ összes eleme igaz $k$ mellett), akkor $k \models \varphi$,
\item $1$-rendben: minden $\mathcal{A}$ struktúrára, ha $\mathcal{A} \models \Sigma$, akkor $\mathcal{A} \models \varphi$.
\end{itemize}
\end{defi}
\begin{exmpl} 
\begin{itemize}
\item $\mathbb{Q}, \mathbb{C}$ racionális és komplex számok gyűrűje mellett legyen $\varphi = \exists x(x\cdot x = -1)$, ekkor $\mathbb{Q} \not\models \varphi$, viszont $\mathbb{C} \models \varphi$.
\item Ha megadunk egy olyan $k$ értékelést, amire $k(x)=5$, akkor a $\mathbb{Q} \models (x = 5)[k]$ igaz lesz, viszont egy másik $l$ értékelés mellett, amire $x$ változóhoz $0$-át rendelünk, azaz $l(x)=0$, már $\mathbb{Q} \not\models (x = 5)[l]$ adódik.
\end{itemize}
\end{exmpl}
\section{előadás}
\begin{defi}[Elősorrend] Ami előrébb van a listán, az köt erősebben: 
$$\frac{\forall}{\exists}, \lnot, \frac{\lor}{\land}, \Rightarrow, \Leftrightarrow.$$
\end{defi}
Például $\exists x \varphi \land \psi$ azt jelenti, hogy $(\exists x \varphi) \land \psi$. Az implikációs láncokat mindig jobbra zárójelezzük: $\varphi \Rightarrow \psi \Rightarrow \rho$ azt jelenti, hogy $\varphi \Rightarrow (\psi \Rightarrow \rho)$.
\subsubsection{Műveletek kapcsolatai}
\begin{defi} Az $f: \lbrace i, h \rbrace^n \rightarrow \lbrace i, h \rbrace$ alakú függvényeket logikai műveleteknek nevezzük. (Az $n$ változós logikai műveletek száma: $2^{2^n}$)
\end{defi}
\begin{defi} Logikai műveletek egy $F$ halmaza funkcionálisan teljes, ha $F$ segítségével az összes $\lbrace i, h \rbrace^n \rightarrow \lbrace i,h \rbrace$ logikai függvény előállítható.
\end{defi}
\begin{theo}
Az $\lbrace \lnot, \land, \lor \rbrace$ művelethalmaz funkcionálisan teljes.
\end{theo}
Most megadunk egy $f$ logikai függvényt (műveletet) igazságtáblázatával.
\begin{displaymath}
\begin{array}{|c|c|c||c}
   x
 & y
 & z
 & f(x,y,z) \\
\hline
h & h & h & h \\
h & h & i & i \\
h & i & h & i \\
h & i & i & h \\
i & h & h & h \\
i & h & i & h \\
i & i & h & h \\
i & i & i & i \\
\hline
\end{array}
\end{displaymath}
Tekintsük a következő formulákat:
\begin{itemize}
\item $\varphi_2 = \lnot x \land \lnot y \land z$
\item $\varphi_3 = \lnot x \land y \land  \lnot z$
\item $\varphi_8 = x \land y \land z$
\end{itemize}
Az egyes formulák csak az indexükben jelzett sorhoz tartozó bemenetre vesznek fel igaz értéket, tehát $\varphi = \varphi_2 \lor \varphi_3 \lor \varphi_8$ megvalósítja $f$-et. \\
\indent Következzen a 3.1 tétel bizonyítása.
\begin{proof}
Legyen $n \in \mathbb{N}$ és $f: \lbrace i, h \rbrace^n \rightarrow \lbrace i, h \rbrace$ tetszőleges, ha $\underline{s} \in \lbrace i, h \rbrace ^n$, akkor legyen:
$$
x_j^{(\underline{s})} =
\begin{cases}
x_j, & \text{ ha $s_j = i$} \\
\lnot x_j, & \text{ ha $s_j = h$}.
\end{cases}
$$
Ekkor
$$\varphi = \bigvee_{\forall \underline{s} : f(\underline{s}) = i} \left( \bigwedge^n_{i=1} x^{(\underline{s})}_i \right),$$
és $f \equiv \varphi$, hiszen $\bigwedge^n_{i=1} x^{(\underline{s})}_i$ mindig csak az adott $\underline{s}$ mellett igaz (az előző példához hasonlóan). Az $f$ függvény ezen előállítását diszjunktív normálformának nevezzük (rövidítve: DNF). \\
\indent Mivel a DNF csak a kitüntetett $\lbrace \lnot, \lor, \land \rbrace$ művelethalmaz logikai függvényeit használja, ezért az valóban funkcionálisan teljes. \qed \\
\indent Egy másik bizonyításhoz definiáljuk a konjunktív normálformát (KNF). Legyen
$$
x_{j(\underline{s})} =
\begin{cases}
\lnot x_j, & \text{ ha $s_j = i$} \\
x_j, & \text{ ha $s_j = h$}
\end{cases},
$$
ekkor
$$\psi = \bigwedge_{\forall \underline{s} : f(\underline{s}) = h} \left( \bigvee^n_{i=1} x^{(\underline{s})}_i \right)$$
az $f$ függvény KNF-es előállítása, ez teljesül, hiszen $\bigvee^n_{i=1} x^{(\underline{s})}_i$ csak a megadott $\underline{s}$ mellett hamis, így $\psi \equiv f$ itt is. A KNF is csak a kitüntetett művelethalmaz elemeit használja. Ezzel befejeztük a másik bizonyítást is.
\end{proof}
\begin{megj}
$\lnot, \land, \lor$ logikai áramkörökkel megvalósíthatók. Ha az áramköri elemeknek költsége van és $f$ kevésszer igaz, akkor DNF, ha kevésszer hamis, akkor KNF alkalmazása a célszerű. A minimális (pl.: legkevesebb logikai függvényt tartalmazó) megvalósítás megkeresése NP-teljes probléma.
\end{megj}
\begin{defi} Legyenek $\varphi$ és $\psi$ formulák. Azt írjuk, hogy $\varphi \equiv \psi$, ha minden $k$ értékelésre $k \models \varphi$ akkor és csak akkor, ha $k \models \psi$.
\end{defi}
\begin{exmpl} $\varphi = \lnot \lnot \varphi$.
\end{exmpl}
\begin{theo} A $\lbrace \lnot, \land \rbrace$ művelethalmaz funkcionálisan teljes.
\end{theo}
\begin{proof} Kell, hogy $\varphi \lor \psi$ kifejezhető $\lbrace \lnot, \land \rbrace$ halmazzal, utána 3.1 tételből következik az állítás.
$\varphi \lor \psi \equiv \lnot \lnot (\varphi \lor \psi) \equiv \lnot (\lnot \varphi \land \lnot \psi)$. Utolsó lépésben egy De Morgan-azonosságot alkalmaztunk.
\end{proof}
\begin{theo} $\lbrace \lnot, \lor \rbrace$ is az.
\end{theo}
\begin{proof} Ugyanúgy mint előbb. $\varphi \land \psi \equiv \lnot(\lnot \varphi \lor \lnot \psi)$.
\end{proof}
\section{előadás}
\subsection{Bizonyításelmélet}
\begin{defi}[Az ítéletkalkulus axióma-sémái]
\begin{minipage}[t]{\linewidth}
\begin{Axiom}
\item $\alpha \Rightarrow (\beta \Rightarrow \alpha)$ \label{axi:one}
\item $(\alpha \Rightarrow (\beta \Rightarrow \gamma)) \Rightarrow ((\alpha \Rightarrow \beta) \Rightarrow (\alpha \Rightarrow \gamma))$
\item $(\lnot \alpha \Rightarrow \beta) \Rightarrow ((\lnot \alpha \Rightarrow \lnot \beta) \Rightarrow \alpha)$,
\end{Axiom}
\end{minipage}
ahol $\alpha, \beta, \gamma$ formulaváltozók.
\end{defi}
\begin{megj}
A $\lbrace \lnot, \Rightarrow \rbrace$ művelethalmaz funkcionálisan teljes és feltesszük, hogy formuláinkban csak ezek fordulnak elő. Mostantól (amíg mást nem mondunk) az ítéletkalkulusban dolgozunk.
\end{megj}
\begin{defi}[Következtetési szabály, Modus Ponens, jel.: $\mathtt{MP}$] $$\frac{\alpha, \alpha \Rightarrow \beta}{\beta}$$ Ez azt jelenti, hogy ha rendelkezésünkre áll $\alpha \Rightarrow \beta$ formula és az $\alpha$, akkor $\mathtt{MP}$-vel előáll $\beta$ is.
\end{defi}
\begin{defi}
Legyen $\Sigma$ formulahalmaz és $\varphi$ formula. Azt írjuk, hogy $\Sigma \vdash \varphi$ (és azt mondjuk, hogy $\Sigma$-ból levezethető $\varphi$), ha van a formuláknak egy olyan véges $\alpha_1, \alpha_2, \ldots, \alpha_n$ sorozata úgy, hogy $\alpha_n = \varphi$ és minden $i \leqslant n$-re $\alpha_i$-re teljesül, hogy:
\begin{itemize}
\item axióma VAGY
\item eleme a $\Sigma$-nak VAGY
\item $\mathtt{MP}$-vel előáll a korábbi $\alpha_j$-kből.
\end{itemize} 
\end{defi}
\begin{megj}
Az előző definícióban szereplő $\alpha_1, \alpha_2, \ldots, \alpha_n$ sorozat $\varphi$ egy $\Sigma$-beli levezetése. Algoritmikusan ellenőrizhető, hogy az adott $\alpha_1, \alpha_2, \ldots, \alpha_n$ levezetése-e $\varphi$-nek.
\end{megj}
\begin{exmpl} $\emptyset \vdash \varphi \Rightarrow \varphi$, ahol $\varphi$ tetszőleges rögzített formula.
\begin{proof}
\begin{tabular}{ l c r }
  1. & $(\varphi \Rightarrow (\varphi \Rightarrow \varphi) \Rightarrow \varphi) \Rightarrow (\varphi \Rightarrow (\varphi \Rightarrow \varphi)) \Rightarrow (\varphi \Rightarrow \varphi)$ &  AX2 egy példánya \\
  2. & $\varphi \Rightarrow (\varphi \Rightarrow \varphi) \Rightarrow \varphi$ & AX1 egy példánya \\
  3. &  $(\varphi \Rightarrow (\varphi \Rightarrow \varphi)) \Rightarrow (\varphi \Rightarrow \varphi)$ & $\texttt{MP}(1,2)$ \\
  4. & $\varphi \Rightarrow (\varphi \Rightarrow \varphi)$ & AX1 egy példánya \\
  5. &  $\varphi \Rightarrow \varphi$ & $\texttt{MP}(3,4)$
\end{tabular}
\end{proof}
\end{exmpl}
\begin{theo}[Teljességi-tétel]
$\Sigma \models \varphi$ akkor és csak akkor, ha $\Sigma \vdash \varphi$, azaz minden helyes következtetés formálisan bizonyítható.
\end{theo}
\begin{megj} $\Sigma \vdash \varphi$ esetén csak szintaktikus transzformációk kerülnek szóba. Jelentés, igazság fel sem merül.
\end{megj}
\begin{defi}[Lezárási operátor] Legyen $S$ halmaz és jelölje $\mathcal{P}(S)$ ennek a hatványhalmazát a $cl : \mathcal{P}(S) \rightarrow \mathcal{P}(S)$ függvényt lezárási operátornak nevezzük az $S$ halmazon, hogy ha teljesülnek az alábbiak minden $X, Y \subseteq S$ halmazra:
\begin{enumerate}
\item $X \subseteq cl(X)$ (extenzív)
\item $X \subseteq Y \Rightarrow cl(X) \subseteq cl(Y)$ (monoton)
\item $cl(cl(X))=cl(X)$ (idempotens)
\end{enumerate}
\end{defi}
\begin{defi} $\ded(\Sigma) := \lbrace \varphi : \Sigma \vdash \varphi \rbrace$, azaz $\ded(\Sigma)$ a $\Sigma$-ból levezethető formulák halmaza.
\end{defi}
\begin{lem} A $\ded$ egy lezárási operátor.
\end{lem}
\begin{proof}
(1) extenzív: Ha $\varphi \in \Sigma$, akkor ''$\varphi$'' levezetése $\varphi$-nek $\Sigma$-ból, tehát $\Sigma \subseteq \ded(\Sigma)$. \\
\indent (2) monoton: Ha $\varphi \in \ded(\Sigma)$, akkor létezik $\alpha_1, \alpha_2, \ldots, \alpha_n$ véges formulasorozat, hogy $\alpha_n=\varphi$ és minden $i \leqslant n$-re $\alpha_i$ 4.3. definíció szerint áll elő, de akkor ugyanez a sorozat egy $\Sigma \subseteq \Gamma$-beli levezetés is, tehát $\ded(\Sigma) \subseteq \ded(\Gamma)$. \\
\indent (3) idempotens: (1) miatt $\Sigma \subseteq \ded(\Sigma)$, tehát (2) miatt $\ded(\Sigma) \subseteq \ded(\ded(\Sigma))$. Másik tartalmazás: legyen $\varphi \in \ded(\ded(\Sigma))$, ekkor létezik olyan $\alpha_1, \alpha_2, \ldots, \alpha_n = \varphi$ formulasorozat, hogy minden $i \leqslant n$-re $\alpha_i$:
\begin{itemize}
\item axióma VAGY
\item eleme $\ded(\Sigma)$-nak VAGY
\item $\texttt{MP}$-vel áll elő a korábbi $\alpha_j$-kből.
\end{itemize} Azonban bármely $\alpha_k \in \ded(\Sigma)$ lecserélhető $\Sigma$-beli levezetésre, ezeket lecserélve $\Sigma \vdash \varphi$ adódik, tehát $\ded(\ded(\Sigma)) \subseteq \ded(\Sigma)$. A kölcsönös tartalmazás miatt az egyenlőség fennáll.
\end{proof}
\begin{theo}[Dedukciós-tétel] Az alábbiak ekvivalensek:
\begin{enumerate}
\item $\Sigma \vdash \varphi \Rightarrow \psi$
\item $\Sigma \cup \lbrace \varphi \rbrace \vdash \psi.$
\end{enumerate}
\end{theo}
\begin{proof}
Először (1) $\Rightarrow$ (2) bizonyítása következik. Tegyük fel, hogy $\Sigma \vdash \varphi \Rightarrow \psi$, ekkor a $\ded$ monotonitása miatt $\Sigma \cup \lbrace \varphi \rbrace \vdash \varphi \Rightarrow \psi$, az extenzivitás miatt pedig $\Sigma \cup \lbrace \varphi \rbrace \vdash \varphi$. Így $\texttt{MP}$-vel elő tud állni $\psi$, azaz $\Sigma \cup \lbrace \varphi \rbrace \vdash \psi$. \\
\indent Most következzen (2) $\Rightarrow$ (1). Tegyük fel, hogy $\alpha_1, \alpha_2, \ldots, \alpha_n$ a $\psi$ levezetése $\Sigma \cup \lbrace \varphi \rbrace$-ből, $i$ szerinti teljes indukcióval belátjuk, hogy $\Sigma \vdash \varphi \Rightarrow \alpha_i$ minden $i \leqslant n$-re. \\
\indent Legyen $i=1$ (3 eset lehetséges), ha $\alpha_1 \in \Sigma$, akkor AX1 alkalmazásával $\Sigma \vdash \alpha_1 \Rightarrow \varphi \Rightarrow \alpha_1$ és mivel $\alpha_1 \in \Sigma$, ezért $\Sigma \vdash \alpha_1$ és így $\varphi \Rightarrow \alpha_1$ $\texttt{MP}$-vel előállítható. \\
\indent Ha $\alpha_1 = \varphi$, akkor 4.1 példa szerint $\emptyset \vdash \varphi \Rightarrow \varphi$, így a monotonitás miatt $\Sigma \vdash \varphi \Rightarrow \varphi$. \\
\indent Utolsó esetben, ha $\alpha_1$ axióma, akkor AX1 alkalmazásával $\Sigma \vdash \alpha_1 \Rightarrow \varphi \Rightarrow \alpha_1$ és mivel $\alpha_1$ axióma, ezért $\Sigma \vdash \alpha_1$ és így $\texttt{MP}$-vel előállítható $\varphi \Rightarrow \alpha_1$, azaz $\Sigma \vdash \varphi \Rightarrow \alpha_1$.\\
\indent Az indukciós lépés: \\
Tegyük fel, hogy minden $j < i$-re tudjuk, hogy $\Sigma \vdash \varphi \Rightarrow \alpha_j$ (kell: $\Sigma \vdash \varphi \Rightarrow \alpha_i$). Amennyiben $\alpha_i$ axióma vagy $\alpha_i \in \Sigma \cup \lbrace \varphi \rbrace$, akkor $i=1$-ben szereplő eljárást kell megismételni. \\
\indent Amennyiben $\alpha_1$ $\texttt{MP}$-vel áll elő, akkor léteznek olyan $\alpha_k, \alpha_l$ formulák, hogy $k,l < i$ és $\alpha_l = (\alpha_k \Rightarrow \alpha_i)$. Az indukciós feltevés miatt $\Sigma \vdash \varphi \Rightarrow \alpha_l$, azaz 
\begin{equation*}
\begin{split}
\Sigma &\vdash \varphi \Rightarrow (\alpha_k \Rightarrow \alpha_i) \\
& \vdash (\varphi \Rightarrow (\alpha_k \Rightarrow \alpha_i)) \Rightarrow (\varphi \Rightarrow \alpha_k) \Rightarrow (\varphi \Rightarrow \alpha_i) \text{ AX2 alkalmazásával} \\
& \vdash (\varphi \Rightarrow \alpha_k) \Rightarrow (\varphi \Rightarrow \alpha_i) \text{ $\texttt{MP}$-vel} \\
& \vdash (\varphi \Rightarrow \alpha_k) \text{ az indukciós feltevés miatt} \\ 
& \vdash \varphi \Rightarrow \alpha_i \text{ $\texttt{MP}$-vel}
\end{split}
\end{equation*}
\end{proof}
\begin{defi}
Legyen $\Sigma$ formulahalmaz. $\Sigma$-t ellentmondásosnak nevezzük, ha van olyan $\alpha$, hogy $\Sigma \vdash \alpha$ és $\Sigma \vdash \lnot \alpha$.
\end{defi}
\begin{lem} Legyen $\Sigma$ formulahalmaz és $\varphi$ egy formula: 1. $\Leftrightarrow$ 2. és 3. $\Leftrightarrow$ 4., ahol
\begin{enumerate}
\item $\Sigma \vdash \varphi$,
\item $\Sigma \cup \lbrace \lnot \varphi \rbrace$ ellentmondásos,
\item $\Sigma \vdash \lnot \varphi$,
\item $\Sigma \cup \lbrace \varphi \rbrace$ ellentmondásos.
\end{enumerate}
\end{lem}
\begin{proof}
1. $\Rightarrow$ 2. Tegyük fel, hogy $\Sigma \vdash \varphi$. A monotonitás miatt $\Sigma \cup \lbrace \lnot \varphi \rbrace \vdash \varphi$. Továbbá $\lnot \varphi \in \Sigma \cup \lbrace \lnot \varphi \rbrace$, így $\Sigma \cup \lbrace \lnot \varphi \rbrace \vdash \lnot \varphi$, azaz $\Sigma \cup \lbrace \lnot \varphi \rbrace$ ellentmondásos. \\
\indent 2. $\Rightarrow$ 1. Tegyük fel, hogy $\Sigma \cup \lbrace \lnot \varphi \rbrace$ ellentmondásos, így létezik olyan $\alpha$, hogy $\Sigma \cup \lbrace \lnot \varphi \rbrace \vdash \alpha$ és $\Sigma \cup \lbrace \lnot \varphi \rbrace \vdash \lnot \alpha$. \\
\indent A dedukciós-tételt alkalmazva az előbbi két esetben adódik, hogy: 
\begin{equation*}
\begin{split}
\Sigma &\vdash \lnot \varphi \Rightarrow \alpha \\
&\vdash \lnot \varphi \Rightarrow \lnot \alpha \\
& \vdash (\lnot \varphi \Rightarrow \alpha) \Rightarrow (\lnot \varphi \Rightarrow \lnot \alpha) \Rightarrow \varphi \text{ AX3 egy példánya} \\
& \vdash \varphi \text{ $\texttt{MP}$ kétszer alkalmazva.}
\end{split}
\end{equation*}
Az utolsó Modus Ponens részletezve: 1. és 3. sorból előáll $(\lnot \varphi \Rightarrow \lnot \alpha) \Rightarrow \varphi$, ebből és 2. sorból előáll $\varphi$. \\
\indent 3. $\Rightarrow$ 4. pontosan úgy, mint 1. $\Rightarrow$ 2. \\
\indent 4. $\Rightarrow$ 3. Tegyük fel, hogy $\Sigma \cup \lbrace \varphi \rbrace$ ellentmondásos. Vegyük észre, hogy $\lbrace \lnot \lnot \varphi, \lnot \varphi \rbrace$ ellentmondásos formulahalmaz, így 2. $\Rightarrow$ 1. miatt $\lnot \lnot \varphi \vdash \varphi$, emiatt $\ded(\Sigma \cup \lbrace \varphi \rbrace) \subseteq \ded(\Sigma \cup \lbrace \lnot \lnot \varphi \rbrace)$ és így $\Sigma \cup \lbrace \lnot \lnot \varphi \rbrace$ is ellentmondásos. Alkalmazva 2. $\Rightarrow$ 1. állítást $\Sigma \vdash \lnot \varphi$, azaz 3. fennáll.
\end{proof}
\begin{theo} Ha $\Sigma$ ellentmondásos és $\varphi$ tetszőleges, akkor $\Sigma \vdash \varphi$.
\end{theo}
\begin{proof} Tegyük fel, hogy $\Sigma$ ellentmondásos, azaz létezik egy olyan $\alpha$, hogy $\Sigma \vdash \alpha$ és $\Sigma \vdash \lnot \alpha$. A monotonitás miatt $\Sigma \cup \lbrace \lnot \varphi \rbrace \vdash \alpha, \lnot \alpha$, amiből $\Sigma \cup \lbrace \lnot \varphi \rbrace$ ellentmondásossága következik. Az előző tétel 2. $\Rightarrow$ 1. része miatt $\Sigma \vdash \alpha$ teljesül.
\end{proof}
\part*{Ajánlott irodalom}
\begin{itemize}
\item Hajnal-Hamburger: Halmazelmélet
\item Csirmaz László: Matematikai logika (angolul: \url{http://eprints.renyi.hu/55/},\\ magyarul: \url{http://eprints.renyi.hu/12/})
\item Várterész Magda: A matematikai logika alkalmazásszemléletű tárgyalása
\item Ferenczi Miklós: Matematikai logika (\url{http://www.math.bme.hu/~ferenczi/LOGIKA.pdf})
\end{itemize}
\end{document}