\begin{defi}[Elősorrend] Ami előrébb van a listán, az köt erősebben: 
$$\frac{\forall}{\exists}, \lnot, \frac{\lor}{\land}, \Rightarrow, \Leftrightarrow.$$
\end{defi}
Például $\exists x \varphi \land \psi$ azt jelenti, hogy $(\exists x \varphi) \land \psi$. Az implikációs láncokat mindig jobbra zárójelezzük: $\varphi \Rightarrow \psi \Rightarrow \rho$ azt jelenti, hogy $\varphi \Rightarrow (\psi \Rightarrow \rho)$.
\subsubsection{Műveletek kapcsolatai}
\begin{defi} Az $f: \lbrace i, h \rbrace^n \rightarrow \lbrace i, h \rbrace$ alakú függvényeket logikai műveleteknek nevezzük. (Az $n$ változós logikai műveletek száma: $2^{2^n}$)
\end{defi}
\begin{defi} Logikai műveletek egy $F$ halmaza funkcionálisan teljes, ha $F$ segítségével az összes $\lbrace i, h \rbrace^n \rightarrow \lbrace i,h \rbrace$ logikai függvény előállítható.
\end{defi}
\begin{theo}
Az $\lbrace \lnot, \land, \lor \rbrace$ művelethalmaz funkcionálisan teljes.
\end{theo}
Most megadunk egy $f$ logikai függvényt (műveletet) igazságtáblázatával.
\begin{displaymath}
\begin{array}{|c|c|c||c}
   x
 & y
 & z
 & f(x,y,z) \\
\hline
h & h & h & h \\
h & h & i & i \\
h & i & h & i \\
h & i & i & h \\
i & h & h & h \\
i & h & i & h \\
i & i & h & h \\
i & i & i & i \\
\hline
\end{array}
\end{displaymath}
Tekintsük a következő formulákat:
\begin{itemize}
\item $\varphi_2 = \lnot x \land \lnot y \land z$
\item $\varphi_3 = \lnot x \land y \land  \lnot z$
\item $\varphi_8 = x \land y \land z$
\end{itemize}
Az egyes formulák csak az indexükben jelzett sorhoz tartozó bemenetre vesznek fel igaz értéket, tehát $\varphi = \varphi_2 \lor \varphi_3 \lor \varphi_8$ megvalósítja $f$-et. \\
\indent Következzen a 3.1 tétel bizonyítása.
\begin{proof}
Legyen $n \in \mathbb{N}$ és $f: \lbrace i, h \rbrace^n \rightarrow \lbrace i, h \rbrace$ tetszőleges, ha $\underline{s} \in \lbrace i, h \rbrace ^n$, akkor legyen:
$$
x_j^{(\underline{s})} =
\begin{cases}
x_j, & \text{ ha $s_j = i$} \\
\lnot x_j, & \text{ ha $s_j = h$}.
\end{cases}
$$
Ekkor
$$\varphi = \bigvee_{\forall \underline{s} : f(\underline{s}) = i} \left( \bigwedge^n_{i=1} x^{(\underline{s})}_i \right),$$
és $f \equiv \varphi$, hiszen $\bigwedge^n_{i=1} x^{(\underline{s})}_i$ mindig csak az adott $\underline{s}$ mellett igaz (az előző példához hasonlóan). Az $f$ függvény ezen előállítását diszjunktív normálformának nevezzük (rövidítve: DNF). \\
\indent Mivel a DNF csak a kitüntetett $\lbrace \lnot, \lor, \land \rbrace$ művelethalmaz logikai függvényeit használja, ezért az valóban funkcionálisan teljes. \qed \\
\indent Egy másik bizonyításhoz definiáljuk a konjunktív normálformát (KNF). Legyen
$$
x_{j(\underline{s})} =
\begin{cases}
\lnot x_j, & \text{ ha $s_j = i$} \\
x_j, & \text{ ha $s_j = h$}
\end{cases},
$$
ekkor
$$\psi = \bigwedge_{\forall \underline{s} : f(\underline{s}) = h} \left( \bigvee^n_{i=1} x^{(\underline{s})}_i \right)$$
az $f$ függvény KNF-es előállítása, ez teljesül, hiszen $\bigvee^n_{i=1} x^{(\underline{s})}_i$ csak a megadott $\underline{s}$ mellett hamis, így $\psi \equiv f$ itt is. A KNF is csak a kitüntetett művelethalmaz elemeit használja. Ezzel befejeztük a másik bizonyítást is.
\end{proof}
\begin{megj}
$\lnot, \land, \lor$ logikai áramkörökkel megvalósíthatók. Ha az áramköri elemeknek költsége van és $f$ kevésszer igaz, akkor DNF, ha kevésszer hamis, akkor KNF alkalmazása a célszerű. A minimális (pl.: legkevesebb logikai függvényt tartalmazó) megvalósítás megkeresése NP-teljes probléma.
\end{megj}
\begin{defi} Legyenek $\varphi$ és $\psi$ formulák. Azt írjuk, hogy $\varphi \equiv \psi$, ha minden $k$ értékelésre $k \models \varphi$ akkor és csak akkor, ha $k \models \psi$.
\end{defi}
\begin{exmpl} $\varphi = \lnot \lnot \varphi$.
\end{exmpl}
\begin{state} A $\lbrace \lnot, \land \rbrace$ művelethalmaz funkcionálisan teljes.
\end{state}
\begin{proof} Kell, hogy $\varphi \lor \psi$ kifejezhető $\lbrace \lnot, \land \rbrace$ halmazzal, utána 3.1 tételből következik az állítás.
$\varphi \lor \psi \equiv \lnot \lnot (\varphi \lor \psi) \equiv \lnot (\lnot \varphi \land \lnot \psi)$. Utolsó lépésben egy De Morgan-azonosságot alkalmaztunk.
\end{proof}
\begin{state} $\lbrace \lnot, \lor \rbrace$ és $\lbrace \lnot, \Rightarrow \rbrace$ is funkcionálisan teljes művelethalmazok.
\end{state}
\begin{proof} Első halmaz: ugyanúgy mint előbb. $\varphi \land \psi \equiv \lnot(\lnot \varphi \lor \lnot \psi)$. Második halmazra: igazságtáblával igazolható, hogy $\varphi \lor \psi \equiv \lnot \varphi \Rightarrow \psi$.
\end{proof}
\begin{fairytale} $\lbrace \lnot \rbrace$ nem funkcionálisan teljes, mert csak egyváltozós függvények írhatóak le vele. A $\lbrace \lor, \land \rbrace$ sem az, igazolható, hogy az ezekkel felírt függvények monoton növőek (ehhez legyen $h < i$), de $\lnot$ nem az, ezért az nem fejezhető ki.
\end{fairytale}
\begin{defi}[Sheffer-féle sem-sem művelet, NAND]
\begin{displaymath}
\begin{array}{|c|c||c}
   x
 & y
 & x \mid y  \\
\hline
h & h & h \\
h & i & h \\
i & i & h \\
h & h & i \\
\hline
\end{array}
\end{displaymath}
\end{defi}
\begin{state}
$\lbrace \mid \rbrace$ funkcionálisan teljes.
\end{state}
\begin{proof}
$\lnot x \equiv x \mid x$ és $x \lor y \equiv \lnot (x \mid y) \equiv (x \mid y) \mid (x \mid y)$, 3.2. állítás miatt ez is funkcionálisan teljes.
\end{proof}
\begin{state}Legyen $\varphi$ tetszőleges formula, ekkor van olyan DNF vagy KNF alakú $\varphi'$, hogy $\varphi \equiv \varphi'$.
\end{state}
\begin{proof} Legyen $t$ a $\varphi$ igazságtáblája 3.1. tétel miatt $\varphi$ leírható $\psi$ KNF-el vagy DNF-el és nyilván $\varphi \equiv \psi$.
\end{proof}
\begin{defi}[Szabadon előforduló változó] A $v$ változó előfordulása $\varphi$-ben szabad, ha az adott előfordulásban nincs $v$-re vonatkozó kvantor hatáskörében. A $v$ szabad $\varphi$-ben, ha $v$-nek van szabad előfordulása $\varphi$-ben.
\end{defi}
\begin{exmpl} $\forall x P(x) \Rightarrow R(x,y)$-ban $y$ szabad előfordulású.  $\forall x \forall y (xy = yx)$ ebben pedig nincs szabad előfordulású változó.
\end{exmpl}
\begin{state} Legyen $t$ egy term, $k$ és $k'$ olyan értékelések, hogy $t$ minden $v$ változójára $k(v) = k'(v)$, akkor $t^\mathcal{A}[k] = t^{\mathcal{A}}[k']$, azaz $t$ jelentése csak a benne előforduló változóktól függ.
\end{state}
\begin{proof}$t$ összetettsége szerinti indukcióval bizonyítunk. 
Ha $t$ konstans szimbólum, akkor nyilvánvalóan $t^\mathcal{A}[k] = t^{\mathcal{A}}[k']$. Ha $t$ változó, akkor $t^\mathcal{A}[k] = k(t) = k'(t) = t^{\mathcal{A}}[k']$. Tegyük fel, hogy $t = f(t_1,t_2,\ldots,t_{n})$ és az állítást már tudjuk $t_1,t_2,\ldots,t_{n}$-re, ekkor $t^\mathcal{A}[k] = f^\mathcal{A}(t^\mathcal{A}_1[k],\ldots,t^\mathcal{A}_n[k]) = f^\mathcal{A}(t^\mathcal{A}_1[k'],\ldots,t^\mathcal{A'}_n[k]) = t^\mathcal{A}[k']$.
\end{proof}
\begin{theo} Legyen $\varphi$ 1.-rendű formula, $\mathcal{A}$ struktúra $k$ pedig egy $\mathcal{A}$ feletti értékelés. $\mathcal{A} \models \varphi[k]$ csak a $\varphi$-ben szabadon előforduló változóktól függ, azaz ha $k'$ egy olyan értékelés, hogy minden $v$ szabad változóra $\varphi$-ben: $k(v)=k'(v)$, akkor $\mathcal{A} \models \varphi[k]$ akkor és csak akkor, ha $\mathcal{A} \models \varphi[k']$.
\end{theo}
\begin{proof}
Ha $\varphi = t_1 = t_2$, akkor $\mathcal{A} \models (t_1 = t_2)[k] \leftrightarrow t^\mathcal{A}_1[k] = t^{\mathcal{A}}_2[k] \leftrightarrow $ 3.5. állítás miatt $t^\mathcal{A}[k']_1 = t_2^\mathcal{A}[k'] \leftrightarrow \mathcal{A} \models (t_1 = t_2)[k']$. Ha $\varphi = R(t_1,\ldots,t_n)$, akkor $\mathcal{A} \models R(t_1,\ldots,t_n)[k] \leftrightarrow (t_1^\mathcal{A}[k],\ldots,t_n^{\mathcal{A}}[k]) \in R^{\mathcal{A}} \leftrightarrow \mathcal{A} \models R(t_1,\ldots,t_n)[k']$, ismét a 3.5. állítást használva. \\
\indent Következzen az indukciós lépés: tegyük fel, hogy az állítás igaz $\varphi$-re, $\psi$-re. Elég megmutatni, hogy az állítás igaz marad $\lnot \varphi$-re, $\varphi \land \psi$-re (3.1. állítás miatt ez így már funkcionálisan teljes) és $\exists v \varphi$-re is. Ha utóbbiak teljesülnek, akkor $\forall v \varphi \equiv \lnot \lnot \forall v \varphi \equiv \lnot (\exists v \lnot \varphi)$ miatt $\forall v \varphi$-re is igaz lesz. Tehát mindennel együtt tetszőleges 1.-rendű formulára igaz lesz. \\
\indent $\varphi$-ről $\lnot \varphi$-re való áttérés:
\begin{equation*}
\begin{split}
\mathcal{A}& \models \lnot \varphi[k] \leftrightarrow \mathcal{A} \not\models \varphi[k] \\
&\stackrel{\text{ind.}}{\leftrightarrow} \\
\mathcal{A}& \not \models \varphi[k'] \leftrightarrow \mathcal{A} \models \lnot \varphi[k'].
\end{split}
\end{equation*}
\indent $\varphi, \psi$-ről $\varphi \land \psi$-re való áttérés:
\begin{equation*}
\begin{split}
\mathcal{A} &\models (\varphi \land \psi)[k] \leftrightarrow \mathcal{A} \models \varphi[k] \text{ és } \mathcal{A} \models \psi[k] \\
&\stackrel{\text{ind.}}{\leftrightarrow} \\
\mathcal{A}& \models \varphi[k'] \text{ és } \mathcal{A} \models \psi[k'] \leftrightarrow \mathcal{A} \models (\varphi \land \psi)[k']. 
\end{split}
\end{equation*}
\indent $\varphi$-ről $\exists v \varphi$-re való áttérés:
$\mathcal{A} \models \exists v \varphi[k] \leftrightarrow$ van $k^* \stackrel{v}{\equiv} k$ értékelés, amelyre $\mathcal{A} \models \varphi[k']$. Legyen
$$k^*'(x) = 
\begin{cases}
k'(x), & \text{ ha } x \neq v \\
k^*(x) & \text{ különben.}
\end{cases}$$
A $\varphi$ szabad változóin $k^*$ és $k^*'$ értékei megegyeznek. Az indukció miatt $\mathcal{A} \models \exists v \varphi[k]$ akkor csak akkor, ha $\mathcal{A} \models \varphi[k^*'] \leftrightarrow \mathcal{A} \models \exists v \varphi[k']$, mert $k' \stackrel{v}{\equiv} k^*'$.
\end{proof}
