\subsection{Bizonyításelmélet}
\begin{defi}[Az ítéletkalkulus axióma-séma]
\begin{minipage}[t]{\linewidth}
\begin{Axiom}
\item $\alpha \Rightarrow (\beta \Rightarrow \alpha)$ \label{axi:one}
\item $(\alpha \Rightarrow (\beta \Rightarrow \gamma)) \Rightarrow ((\alpha \Rightarrow \beta) \Rightarrow (\alpha \Rightarrow \gamma))$
\item $(\lnot \alpha \Rightarrow \beta) \Rightarrow ((\lnot \alpha \Rightarrow \lnot \beta) \Rightarrow \alpha)$,
\end{Axiom}
\end{minipage}
ahol $\alpha, \beta, \gamma$ formulaváltozók.
\end{defi}
\begin{megj}
A $\lbrace \lnot, \Rightarrow \rbrace$ művelethalmaz funkcionálisan teljes és feltesszük, hogy formuláinkban csak ezek fordulnak elő. Mostantól (amíg mást nem mondunk) az ítéletkalkulusban dolgozunk.
\end{megj}
\begin{defi}[Következtetési szabály, Modus Ponens, jel.: $\mathtt{MP}$] $$\frac{\alpha, \alpha \Rightarrow \beta}{\beta}$$ Ez azt jelenti, hogy ha van egy $\alpha \Rightarrow \beta$ formulánk és rendelkezésünkre áll $\alpha$, akkor $\mathtt{MP}$-vel előáll $\beta$ is.
\end{defi}
\begin{defi}
Legyen $\Sigma$ formulahalmaz és $\varphi$ formula. Azt írjuk, hogy $\Sigma \vdash \varphi$ (és azt mondjuk, hogy $\Sigma$-ból levezethető $\varphi$), ha van a formuláknak egy olyan véges $\alpha_1, \alpha_2, \ldots, \alpha_n$ sorozata úgy, hogy $\alpha_n = \varphi$ és minden $i \leqslant n$-re $\alpha_i$-re teljesül, hogy:
\begin{itemize}
\item axióma VAGY
\item eleme a $\Sigma$-nak VAGY
\item $\mathtt{MP}$-vel előáll a korábbi $\alpha_j$-kből.
\end{itemize} 
\end{defi}
\begin{megj}
Az előző definícióban szereplő $\alpha_1, \alpha_2, \ldots, \alpha_n$ $\varphi$ egy $\Sigma$-beli levezetése. Algoritmikusan ellenőrizhető, hogy az adott $\alpha_1, \alpha_2, \ldots, \alpha_n$ bizonyítása-e $\varphi$-nek.
\end{megj}
\begin{exmpl} $\emptyset \vdash \varphi \Rightarrow \varphi$, ahol $\varphi$ tetszőleges rögzített formula.
\begin{proof}
\begin{tabular}{ l c r }
  1. & $(\varphi \Rightarrow (\varphi \Rightarrow \varphi)) \Rightarrow (\varphi \Rightarrow (\varphi \Rightarrow \varphi)) \Rightarrow (\varphi \Rightarrow \varphi)$ &  AX2 egy példánya \\
  2. & $\varphi \Rightarrow (\varphi \Rightarrow \varphi) \Rightarrow \varphi$ & AX1 egy példánya \\
  3. &  $(\varphi \Rightarrow (\varphi \Rightarrow \varphi)) \Rightarrow (\varphi \Rightarrow \varphi)$ & $\texttt{MP}(1,2)$ \\
  4. & $\varphi \Rightarrow (\varphi \Rightarrow \varphi)$ & AX1 egy példánya \\
  5. &  $\varphi \Rightarrow \varphi$ & $\texttt{MP}(3,4)$
\end{tabular}
\end{proof}
\end{exmpl}
\begin{theo}[Teljességi-tétel]
$\Sigma \models \varphi$ akkor és csak akkor, ha $\Sigma \vdash \varphi$, azaz minden következtetés formálisan bizonyítható.
\end{theo}
\begin{megj} $\Sigma \vdash \varphi$ esetén csak szintaktikus transzformációk kerülnek szóba. Jelentés, igazság fel sem merül.
\end{megj}
\begin{defi}[Lezárási operátor] Legyen $S$ halmaz és jelölje $\mathcal{P}(S)$ ennek a hatványhalmazát a $cl : \mathcal{P}(S) \rightarrow \mathcal{P}(S)$ függvényt lezárási operátornak nevezzük az $S$ halmazon, hogy ha teljesülnek az alábbiak minden $X, Y \subseteq S$ halmazra:
\begin{enumerate}
\item $X \subseteq cl(X)$ (extenzív)
\item $X \subseteq Y \Rightarrow cl(X) \subseteq cl(Y)$ (monoton)
\item $cl(cl(X))=cl(X)$ (idempotens)
\end{enumerate}
\end{defi}
\begin{defi} $\ded(\Sigma) := \lbrace \varphi : \Sigma \vdash \varphi \rbrace$, azaz $\ded(\Sigma)$ a $\Sigma$-ból levezethető formulák halmaza.
\end{defi}
\begin{lem} A $\ded$ egy lezárási operátor.
\end{lem}
\begin{proof}
(1) extenzív: Ha $\varphi \in \Sigma$, akkor ''$\varphi$'' levezetése $\varphi$-nek $\Sigma$-ból, tehát $\Sigma \subseteq \ded(\Sigma)$. \\
\indent (2) monoton: Ha $\varphi \in \ded(\Sigma)$, akkor létezik $\alpha_1, \alpha_2, \ldots, \alpha_n$ véges formulasorozat, hogy $\alpha_n=\varphi$ és minden $i \leqslant n$-re $\alpha_i$ 4.3. definíció szerint áll elő, de akkor ezek egy $\Sigma \subseteq \Gamma$-ban is benne vannak, tehát $\ded(\Sigma) \subseteq \ded(\Gamma)$. \\
\indent (3) idempotens: (1) miatt $\Sigma \subseteq \ded(\Sigma)$, tehát (2) miatt $\ded(\Sigma) \subseteq \ded(\ded(\Sigma))$. Másik tartalmazás: legyen $\varphi \in \ded(\ded(\Sigma))$, ekkor létezik olyan $\alpha_1, \alpha_2, \ldots, \alpha_n = \varphi$ formulasorozat, hogy minden $i \leqslant n$-re $\alpha_i$:
\begin{itemize}
\item axióma VAGY
\item eleme $\ded(\Sigma)$-nak VAGY
\item $\texttt{MP}$-vel áll elő a korábbi $\alpha_j$-kből.
\end{itemize} Azonban bármely $\alpha_k \in \ded(\Sigma)$ lecserélhető $\Sigma$-beli levezetésre, ezeket lecserélve $\Sigma \vdash \varphi$ adódik, tehát $\ded(\ded(\Sigma)) \subseteq \ded(\Sigma)$. A kölcsönös tartalmazás miatt az egyenlőség fennáll.
\end{proof}
\begin{theo}[Dedukciós-tétel] Az alábbiak ekvivalensek:
\begin{enumerate}
\item $\Sigma \vdash \varphi \Rightarrow \psi$
\item $\Sigma \cup \lbrace \varphi \rbrace \vdash \psi.$
\end{enumerate}
\end{theo}
\begin{proof}
Először (1) $\Rightarrow$ (2) bizonyítása következik. Tegyük fel, hogy $\Sigma \vdash \varphi \Rightarrow \psi$, ekkor a $\ded$ monotonitása miatt $\Sigma \cup \lbrace \varphi \rbrace \vdash \varphi \Rightarrow \psi$, az extenzivitás miatt pedig $\Sigma \vdash \varphi$. Így $\texttt{MP}$-vel elő tud állni $\psi$, azaz $\Sigma \vdash \psi$. \\
\indent Most következzen (2) $\Rightarrow$ (1). Tegyük fel, hogy $\alpha_1, \alpha_2, \ldots, \alpha_n$ a $\psi$ levezetése $\Sigma \cup \lbrace \varphi \rbrace$-ből. Egy $i$ szerinti teljes indukcióval belátjuk, hogy $\Sigma \vdash \varphi \Rightarrow \alpha_i$. \\
\indent Legyen $i=1$ (3 eset lehetséges), ha $\alpha_1 \in \Sigma$, akkor AX1 alkalmazásával $\Sigma \vdash \alpha_1 \Rightarrow \varphi \Rightarrow \alpha_1$ és mivel $\alpha_1 \in \Sigma$, ezért $\Sigma \vdash \alpha_1$ és így $\varphi \Rightarrow \alpha_1$ $\texttt{MP}$-vel előállítható. \\
\indent Ha $\alpha_1 = \varphi$, akkor 4.1 példa szerint $\emptyset \vdash \varphi \Rightarrow \varphi$, így a monotonitás miatt $\Sigma \vdash \varphi \Rightarrow \varphi$. \\
\indent Utolsó esetben, ha $\alpha_1$ axióma, akkor AX1 alkalmazásával $\Sigma \vdash \alpha_1 \Rightarrow \varphi \Rightarrow \varphi_1$ és mivel $\alpha_1$ axióma, ezért $\Sigma \vdash \alpha_1$ és így $\texttt{MP}$-vel előállítható $\varphi \Rightarrow \alpha_1$, azaz $\Sigma \vdash \varphi \Rightarrow \alpha_1$.\\
\indent Az indukciós lépés: \\
Tegyük fel, hogy minden $j < i$-re tudjuk, hogy $\Sigma \vdash \varphi \Rightarrow \alpha_j$ (kell: $\Sigma \vdash \varphi \Rightarrow \alpha_i$). Amennyiben $\alpha_i$ axióma vagy $\alpha_i \in \Sigma \cup \lbrace \varphi \rbrace$, akkor $i=1$-ben szereplő eljárást kell megismételni. \\
\indent Amennyiben $\alpha_1$ $\texttt{MP}$-vel áll elő, akkor léteznek olyan $\alpha_k, \alpha_l$ formulák, hogy $k,l < i$ és $\alpha_l = (\alpha_k \Rightarrow \alpha_i)$. Az indukciós feltevés miatt $\Sigma \vdash \varphi \Rightarrow \alpha_l$, azaz 
\begin{equation*}
\begin{split}
\Sigma &\vdash \varphi \Rightarrow (\alpha_k \Rightarrow \alpha_i) \\
& \vdash (\varphi \Rightarrow (\alpha_k \Rightarrow \alpha_i)) \Rightarrow (\varphi \Rightarrow \alpha_k) \Rightarrow (\varphi \Rightarrow \alpha_i) \text{ AX2 alkalmazásával} \\
& \vdash (\varphi \Rightarrow \alpha_k) \Rightarrow (\varphi \Rightarrow \alpha_i) \text{ újból AX2} \\
& \vdash \varphi \Rightarrow \alpha_i \text{ $\texttt{MP}$-vel első és harmadik sorból.}
\end{split}
\end{equation*}
Ezzel beláttuk, ami kellett.
\end{proof}
\begin{defi}
Legyen $\Sigma$ formulahalmaz. $\Sigma$-t ellentmondásosnak nevezzük, ha van olyan $\alpha$, hogy $\Sigma \vdash \alpha$ és $\Sigma \vdash \lnot \alpha$.
\end{defi}
\begin{lem} Legyen $\Sigma$ formulahalmaz és $\varphi$ egy formula: 1. $\Leftrightarrow$ 2. és 3. $\Leftrightarrow$ 4., ahol
\begin{enumerate}
\item $\Sigma \vdash \varphi$,
\item $\Sigma \cup \lbrace \lnot \varphi \rbrace$ ellentmondásos,
\item $\Sigma \vdash \lnot \varphi$,
\item $\Sigma \cup \lbrace \varphi \rbrace$ ellentmondásos.
\end{enumerate}
\end{lem}
\begin{proof}
1. $\Rightarrow$ 2. Tegyük fel, hogy $\Sigma \vdash \varphi$. A monotonitás miatt $\Sigma \cup \lbrace \lnot \varphi \rbrace \vdash \varphi$. Továbbá $\lnot \varphi \in \Sigma \cup \lbrace \lnot \varphi \rbrace$, így $\Sigma \cup \lbrace \lnot \varphi \rbrace \vdash \lnot \varphi$, azaz $\Sigma \cup \lbrace \lnot \varphi \rbrace$ ellentmondásos. \\
\indent 2. $\Rightarrow$ 1. Tegyük fel, hogy $\Sigma \cup \lbrace \lnot \varphi \rbrace$ ellentmondásos, így létezik olyan $\alpha$, hogy $\Sigma \cup \lbrace \lnot \varphi \rbrace \vdash \alpha$ és $\Sigma \cup \lbrace \lnot \varphi \rbrace \vdash \lnot \alpha$. \\
\indent A dedukciós-tételt alkalmazva az előbbi két esetben adódik, hogy: 
\begin{equation*}
\begin{split}
\Sigma &\vdash \lnot \varphi \Rightarrow \alpha \\
&\vdash \lnot \varphi \Rightarrow \lnot \alpha \\
& \vdash (\lnot \varphi \Rightarrow \alpha) \Rightarrow (\lnot \varphi \Rightarrow \lnot \alpha) \Rightarrow \varphi \text{ AX3 egy példánya} \\
& \vdash \varphi \text{ $\texttt{MP}$ kétszer alkalmazva.}
\end{split}
\end{equation*}
Az utolsó Modus Ponens részletezve: 1. és 3. sorból előáll $(\lnot \varphi \Rightarrow \lnot \alpha) \Rightarrow \varphi$ és a most kinyert formulából és 2. sorból előáll $\varphi$. \\
\indent 3. $\Rightarrow$ 4. pontosan úgy, mint 1. $\Rightarrow$ 2. \\
\indent 4. $\Rightarrow$ 3. Tegyük fel, hogy $\Sigma \cup \lbrace \varphi \rbrace$ ellentmondásos. Vegyük észre, hogy $\lbrace \lnot \lnot \varphi, \lnot \varphi \rbrace$ ellentmondásos formulahalmaz, így 2. $\Rightarrow$ 1. miatt $\lnot \lnot \varphi \vdash \varphi$, emiatt $\ded(\Sigma \cup \lbrace \varphi \rbrace) \subseteq \ded(\Sigma \cup \lbrace \lnot \lnot \varphi \rbrace)$ és így $\Sigma \cup \lbrace \lnot \lnot \varphi \rbrace$ is ellentmondásos. Alkalmazva 2. $\Rightarrow$ 1. állítást $\Sigma \vdash \lnot \varphi$, azaz 3. fennáll.
\end{proof}
\begin{theo} Ha $\Sigma$ ellentmondásos és $\varphi$ tetszőleges, akkor $\Sigma \vdash \varphi$.
\end{theo}
\begin{proof} Tegyük fel, hogy $\Sigma$ ellentmondásos, azaz létezik egy olyan $\alpha$, hogy $\Sigma \vdash \alpha$ és $\Sigma \vdash \lnot \alpha$. A monotonitás miatt $\Sigma \cup \lbrace \lnot \varphi \rbrace \vdash \alpha, \lnot \alpha$, amiből $\Sigma \cup \lbrace \lnot \varphi \rbrace$ ellentmondásossága következik. Az előző tétel 2. $\Rightarrow$ 1. része miatt $\Sigma \vdash \alpha$ teljesül.
\end{proof}